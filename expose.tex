\documentclass[
  paper=a4,
  parskip=half,
  fontsize=12pt,
  listof=toc,
  titlepage,
  headsepline,
  footsepline,
]{scrartcl}

% enable UTF-8 text input
\usepackage[utf8]{inputenc}

% DOCUMENT METADATA

\author{Kevin Sieverding}
\title{AI Agents in Business Software}
\subject{Exposé}

% PREAMBLE

%!TeX root = ../index.tex

% BASICS

% make 3rd party packages behave well with scrartcl
\usepackage{scrhack}

% simple arithmetics
\usepackage{calc}

% language support
\usepackage[american]{babel}

% advanced support for quotations
\usepackage[
  autostyle % adapt quote style to document language
]{csquotes}

% color support
\usepackage[dvipsnames,table,hyperref,fixinclude]{xcolor}

% graphics support
\usepackage[]{graphicx}

% extend file name support for graphicx package
\usepackage[
  extendedchars,
  encoding,
  multidot,
  space,
  filenameencoding=utf8,
]{grffile}

% hyperlink support
\usepackage[
  breaklinks,
  pdfusetitle
]{hyperref}

% bibliography support
\usepackage[
  backend=biber,
  style=authoryear,
  labeldate=year,
  sortcites=true,
  sorting=nyt,
  block=space,
  maxnames=2,
  minnames=1
]{biblatex}

% glossaries and acronyms
\usepackage[
    acronym,
    toc=true,
]{glossaries}

% FORMATTING

% micro-typography support (makes most fonts look a little nicer)
\usepackage{microtype}

% support for controlling line spacing
\usepackage{setspace}

% format captions
\usepackage[font=small,labelfont=bf]{caption}

% easily control widow and club penalties
\usepackage[
  defaultlines=4, % keep at least x lines before/after a page break
  all, % apply to whole document
]{nowidow}

% customize list environments
\usepackage{enumitem}

% support for code listings
\usepackage{minted}

% wrapt text around figures
\usepackage{wrapfig}

% MISC

% useful blindtext in document language
\usepackage{blindtext}

% support for placing todo notes in margins
\usepackage[obeyFinal,color=Red,textsize=tiny]{todonotes}


%!TeX root = ../../index.tex

\usepackage[
  %showframe, % draw the layout frame for "debugging"
  top=3cm,
  bottom=2cm,
  left=3.5cm,           % 3.5 + 0.5 Bindingoffset
  right=3cm,
  bindingoffset=0.5cm,
  headheight=22pt,
  headsep=1.25cm-\headheight, % pagetop to header = 1.25cm
  % footheight=\baselineskip,
  footskip=0.75cm, % pagebottom to footer = 1.25cm
  marginparsep=0.3cm,
  marginpar=3.5cm-\marginparsep,
]{geometry}

\setlength{\footheight}{22pt}

%!TeX root=../../index.tex

% include chapter/section titles in headers
\usepackage[
    automark,
]{scrlayer-scrpage}


% Header
\chead[]{}
\ihead[]{}
\ohead[]{\rightmark}

% Footer
\cfoot[]{}
\ifoot[]{}
\ofoot[\pagemark]{\pagemark}

%!TeX root=../../index.tex

%  make \paragraph and \subparagraph behave like small headings
\RedeclareSectionCommands[
    beforeskip=-1sp,
    afterskip=1sp,
    %indent=0pt
]{paragraph,subparagraph}

% Use numbers also for level 1 nested enumerations
\renewcommand{\labelenumii}{\theenumii}
\renewcommand{\theenumii}{\theenumi.\arabic{enumii}.}

%!TeX root = ../index.tex

% support modern fonts
\usepackage[T1]{fontenc}

% sans font that looks similar to Arial
\usepackage{helvet}

% serif font that looks similar to Times New Roman
\usepackage{mathptmx}

% !TeX root = ../../index.tex

\setminted{
  baselinestretch=1.2,
  fontsize=\footnotesize,
  breaklines,
  frame=lines,
  framesep=2mm,
  tabsize=2,
  autogobble
}


% setup hyperref after all packages have been loaded
\hypersetup{
  colorlinks=true,
  allcolors=black,
}

% setup acronyms
% !TeX root = ../index.tex

% \newacronym{cpu}{CPU}{Central Processing Unit}
% \glsunset{cpu}

\newacronym{rag}{RAG}{Retrieval-Augmented Generation}


\makenoidxglossaries{}

% setup bibliography
\addbibresource{assets/bibliography.bib}

% set base path for graphics
\graphicspath{{assets/img}}

% DOCUMENT

\begin{document}

% FRONTMATTER

\maketitle

\clearpage

\newcounter{frontpagecount}
\addtocounter{frontpagecount}{\value{page}}

\clearpage

% CONTENT

% use arabic numerals for numbering content pages
\pagenumbering{arabic}

% use the default header style
\pagestyle{scrheadings}

% count content pages from 1
\setcounter{page}{1}

% MS Word does line spacing wrong.
% To get true one-half spacing use '\onehalfspacing'.
% To get something similar to MS Word, use '\setstretch{1.5}'.
\setstretch{1.5}

Since the launch of ChatGPT, researchers and engineers have invested in bulding a new generation of AI applications beyond the chat-based information retreival using the \gls{rag} pattern: AI Agents.
\parencite{mastermanLandscapeEmergingAI2024}

Conceptually, \cite{wooldridgeIntelligentAgentsTheory1995} define an Agent as a computer system that is capable of acting without direct human intervention (\textit{autonomy}), able to interact with other agents or humans (\textit{social ability}), capable of perceiving its environment (\textit{reactivity}), and exibits goal-directed behavior (\textit{proactiveness}).
More concretely, in the conteporary context of large language models, \cite{mastermanLandscapeEmergingAI2024} define agents as \enquote{language model-powered entities able to plan and take actions to execute goals over multiple iterations. AI Agent architectures are either comprised of a single Agent or multiple agents working together to solve a problem.}

One particular area of interest for AI Agents is enterprise software.
With the constant efforts to increase productivity through automation, the integration of AI Agents appears to be an attractive undertaking.
Notable providers such as Microsoft with Copilot \parencite{microsoftMicrosoftCopilotYour}, Salesforce with Agentforce \parencite{salesforceinc.AgentforceCreatePowerful}, SAP with Business AI \parencite{sapseSAPBusinessAI}, and ServiceNow with their AI Agent offerings \parencite{servicenowAIAgents} have already crowded into the market.
Gartner predicts that 
\enquote{By 2028, [\ldots] 33\% of enterprise software applications will include agentic AI, up from less than 1\% in 2024, with at least 15\% of day-to-day work decisions being made autonomously through AI agents.} \parencite{gartnerinc.HowImplementAI2025}

\cite{mckinsey_why-ai-agents} offer hypothetical use cases for AI Agents in business software, such as:
\begin{description}[font=\small\bfseries]
  \item[Loan underwriting] A set of specialized agents could handle the process of collecting, analyzing and compiling data from various sources to judge the risk of a loan application.
  \item[Code documentation and modernization] A set of specialized agents could analyze, document and document legacy code, or modernize it to a new programming language.
  \item[Online marketing capaign creation] A set of specialized agents could analyze the target audience, create a marketing strategy, and generate the content for the campaign.
\end{description}
In all of these examples, the end user would trigger the process by defining a goal, boundary conditions and a high-level action plan in natural language to kick off the process.

The proposed work aims to investigate the potential of AI Agents in business software.
The focus will be on the following questions:

\begin{itemize}
  \item What are the current capabilities of AI Agents?
  \item Which technologies exist to develop AI Agents?
  \item How can AI Agents be integrated into existing business software?
  \item What are the potential benefits and challenges of using AI Agents in business software?
  \item How can the value of AI Agents be measured and evaluated?
  \item What are the determining factores for end users to perceive AI Agents as valuable?
\end{itemize}

To that end, the work will include a literature review of existing research on AI Agents and their applications in business software.
The literature review will be complemented by a survey of existing technologies and service offerings for developing AI Agents.
Furthermore, if the scope allows, a prototype will be developed to demonstrate the integration of AI Agents into business software.
Using the prototype, a user study will be conducted to evaluate AI Agents in a business context as well as to determine the factors that influence user perception of AI Agents.

% BACKMATTER

\clearpage

% use roman numerals to number backmatter pages
\pagenumbering{Roman}

% use plain headers for backmatter pages
\pagestyle{plain.scrheadings}

% start numbering backmatter pages from 1
\setcounter{page}{\value{frontpagecount}}

% \pagestyle{empty}

% {
%   \centering\Huge\mbox{}
%   \vfill{}
%   Appendix\\
%   \vfill{}
% }

% \clearpage

\pagestyle{plain.scrheadings}

\appendix

% insert appendix sections here

\clearpage

% print bibliography in sloppypar environment for better URL placement
\begin{sloppypar}
  \printbibliography[
    heading=bibintoc % include bibliography in table of contents
  ]{}
\end{sloppypar}

\clearpage

\end{document}